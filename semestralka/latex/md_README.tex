Autor\+: Terezie Hrubanová

\subsection*{Téma z Progtestu}

Vytvořte aplikaci implementující maticovou kalkulačku.

Kalkulačka musí implementovat\+:
\begin{DoxyEnumerate}
\item práci s maticemi různé velikosti
\item základní operace\+: sčítání, odčítání, násobení, transpozice
\item sloučení (3x3 sloučeno s 3x3 =$>$ matice 6x3) a oříznutí (např. pouze levá polovina matice 6x3 =$>$ levá 3x3)
\item pro čtvertcové matice navíc výpočet inverze
\item efektivní paměťovou reprezentaci (pro řídké matice se nehodí 2D pole, pro husté matice naopak)
\item funkce pro výpočet determinantu, určení hodnosti, Gaussova eliminační metoda
\item proměnné (uložení matice do proměnné, načítání matice ze vstupu, využití proměnné ve výpočtu)
\end{DoxyEnumerate}

Použití polymorfismu (doporučeně)


\begin{DoxyItemize}
\item typ matice\+: proměnná, řídká (mapa, list) vs. hustá (2D pole)
\item operace\+: sčítání, odčítání, násobení, ...
\item varianta G\+E\+Mu\+: klasická, s komentářem
\end{DoxyItemize}

Ukázka běhu programu (není závazná)\+: \begin{DoxyVerb}SCAN X[3][3]
Y[3][3] = 1 // jednotkova matice
Z = MERGE X Y
GEM Z -v // s detaily
X = SPLIT Z [3][3] (3, 0) // z matice Z, chci jen 3x3 od pozice (3, 0)
PRINT X
\end{DoxyVerb}


\subsection*{Zadání}

Uživatel může pomocí jednoduchých příkazů načíst matice do proměnných a provádět na nich operace dle zadání. Slovo příkazu je následováno dvojtečkou, a pak jmény proměnných případně jinými parametry, s kterými počítá, příkaz je zakončen mezerou a středníkem.


\begin{DoxyItemize}
\item {\ttfamily load\+:pepa 2 2$\vert$1 2 3 4 ;} vytvoří novou matici 2x2, po řádcích načte její obsah a uloží ji do proměnné \char`\"{}pepa\char`\"{}
\item {\ttfamily print\+:pepa ;} vypíše obsah proměnné \char`\"{}pepa\char`\"{}
\item {\ttfamily add\+:pepa franta ;} sečte existující proměnné \char`\"{}pepa\char`\"{} a \char`\"{}franta\char`\"{} a vypíše součet
\item {\ttfamily subtract\+:pepa franta ;} totéž co add, ale odečítá
\item {\ttfamily multiply\+:pepa franta ;} totéž co subtract, ale násobí matice (v pořadí v jakém byly zadány)
\item {\ttfamily transpose\+:lojza ;} transponuje matici uloženou v proměnné \char`\"{}lojza\char`\"{}
\item {\ttfamily cut\+:franta 1 1$\vert$0 1 ;} ořízne matici v proměnné \char`\"{}franta\char`\"{} na velikost 1x1 od pozice \mbox{[}0\mbox{]}\mbox{[}1\mbox{]}
\item {\ttfamily put\+:henry command\+:variable variable ;} do proměnné \char`\"{}henry\char`\"{} uloží výsledek vnořeného příkazu
\item {\ttfamily help ;} zobrazí nápovědu
\item {\ttfamily exit ;} ukončí program
\item další příkazy podle zadání (inverze, hodnost, determinant atd.)
\end{DoxyItemize}

Běh programu může vypadat třeba takto (vstup od uživatele označen \$)\+:


\begin{DoxyCode}
1 $load:moric 2 2|0 0 0 1 ;
2 Matrix moric successfully loaded.
3 0.000000 0.000000
4 0.000000 1.000000
5 $load:kilian 2 2|0 0 0 0 ;**
6 Matrix kilian successfully loaded.
7 0.000000 0.000000
8 0.000000 0.000000
9 $multiply:moric kilian ;**
10 Result of multiplication is:
11 0.000000 0.000000
12 0.000000 0.000000
\end{DoxyCode}


O běh programu se stará {\ttfamily C\+Application}, o věci specifické pro konzolovou aplikaci (parsování vstupu) pak její potomek {\ttfamily C\+Console\+Application}. Příkazy jsou reprezentovány potomky abstraktní třídy {\ttfamily C\+Command} ({\ttfamily C\+Command\+Add}, {\ttfamily C\+Command\+Load}, ...). Matice reprezentuji dvojím způsobem, podle toho, zda jsou řídké (mapa) nebo husté (2D vector). Jsou to třídy {\ttfamily C\+Matrix\+Sparse} a {\ttfamily C\+Matrix\+Standard} se společným abstraktním předkem {\ttfamily C\+Matrix}.

\subsubsection*{Kde mám polymorfismus?}

Polymorfismus využívám u matic a operací s nimi. {\ttfamily C\+Matrix\+Sparse} a {\ttfamily C\+Matrix\+Standard} mají různě implementované metody na sčítání, odčítání atd., v aplikaci tedy pracuji s pointery na naddtřídu {\ttfamily C\+Matrix}, na kterých se pak polymorfně tyto metody volají.

Polymorfismus je také mezi potomky třídy {\ttfamily C\+Command}, kteří se liší v konkrétním provedení metody {\ttfamily Execute()}. 